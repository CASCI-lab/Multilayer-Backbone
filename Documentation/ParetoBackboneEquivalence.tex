\documentclass{article} 
\usepackage{amsmath}
\usepackage{amsthm}

\newtheorem{theorem}{Theorem}[section]
\newtheorem{corollary}{Corollary}[theorem]
\newtheorem{lemma}[theorem]{Lemma}

\title{Pareto Backbone Results} 
\author{Jordan Rozum} 
\date{\today} % Sets date for date compiled

% The preamble ends with the command \begin{document}
\begin{document} % All begin commands must be paired with an end command somewhere
\maketitle % creates title using information in preamble (title, author, date)

\section{Terminology} % creates a section

Consider weighted digraphs $\left(G_0,G_1,\ldots,G_m\right)$ with equal vertex sets and corresponding
to layers of a multigraph $\mathcal{G}$. A node belonging to layer $l$ is denoted by a superscript, e.g.,
$u^l$ is
node $u$ of $G_l$. A path is a sequence of edges in the multigraph $P_i={(u_i^{l_i},v_i^{l_i})}_i$,
where $v_i=u_{i+1}$. The weight of an edge is determined by the weight function $w$ that sums the
(strictly positive) weights of the edges along an input path.

The {\bf multidistance} of a path is the tuple
\[M(P)=\left(\sum_{i:l_i=0} w(P_i),\sum_{i:l_i=1} w(P_i),\ldots,\sum_{i:l_i=m} w(P_i)\right).\]

We abbreviate this $M(P)=(M_0,M_1,\ldots,M_m)$. We define the partial order $\prec$
on multidistances. We say that $M(P)\prec M(P^\prime)$ if $M_l\leq {M^\prime}_l$ for all $l$ and if
there is at least one value of $l$ for which the inequality is strict; in this case, we say that
$P^\prime$ \emph{dominates} $P$. If neither
$M(P)\prec M(P^\prime)$ nor $M(P^\prime)\prec M(P)$, then the two multidistances are incomparable, and
we write $M(P)\sim M(P^\prime)$. We extend these comparisons to $P$ for notational brevity, e.g.,
$P\prec P^\prime$ is equivalent to $M(P)\prec M(P^\prime)$.

A path $P$ is {\bf pareto minimal} (or shortest) if does not dominate any paths with the same
endpoints. A pareto minimal path's multidistance is \emph{pareto efficient}. Note that there can be
multiple pareto shortest paths with distinct multidistances; these
multidistances will be incomparable, and the set of such multidistances is called the {\bf pareto distance
        set} between the endpoints.

A multidistance can be converted into a traditional distance by computing a weighted sum of its
its entries. The weights correspond to a weighting of the layers. For weight vector $W$,
the distance of a path is

\[
    dist_W(P)=\sum_{l=0}^{m} M_l W_l,
\]

where $M=M(P)$. We assume that $W_l>0$ for all $l$.

Note that if a path $P$ between two nodes is a shortest path (i.e., has minimal distance) for some
weight $W$, then $M(P)$
is an element of the pareto distance set. Otherwise there is a path $P^\prime$ between the same
endpoints with $M_l^\prime<M_l$ for some $l$ and no $j$ such that $M_j<M_j^\prime$, and thus
$dist_W(P^\prime)<dist_W(P)$  (recall that $W_l>0$ for all $l$ by assumption).

However, the reverse is not true. For example, consider the four pareto minimal path multidistances:

\begin{align*}
    M^0=(2,1,1) \\
    M^1=(1,2,1) \\
    M^2=(1,1,2) \\
    M^3=(3,0,0)
\end{align*}


It is straightforward to show that while all four multidistances are incomparable, the first, $M^0$,
cannot yield a minimal distance. To see this, consider an arbitrary weight vector $W=(a,b,c)$ with
$a,b,c>0$. For $M^0$ to be minimal, it must satisfy
\begin{align*}
    2a+b+c \leq a+2b+c \\
    2a+b+c \leq a+b+2c \\
    2a+b+c \leq 3a+0b+0c,
\end{align*}
but the first two inequalities imply $a\leq b,c$, while the last implies $b+c\leq a$. These are
impossible to satisfy simultaneously. It remains true, however, that $M^0$ can be shorter than any
\emph{individual} path under a carefully chosen weighting. No such weighting can make the corresponding
path shorter than all others.

The multidistance $M^3$ in this example, however, illustrates an important special case: its path
is confined to a single layer. Because of this, the pareto shortest path criterion reduces to the
requirement that no other multidistance has a smaller value in the entry corresponding to that layer.
Thus, a weighting can always be chosen to make such paths have minimal distance. To do so, simply assign a very
small weight to that layer, and a very large weight to all others. We formalize this as a lemma:

\begin{lemma}
    Let a path $P$ have a multidistance $M(P)$ that is zero in all but one entry. Then if $P$ is an
    element of the pareto distance set between two nodes, there is a layer weighting $W$ for which
    $dist_W(P)$ is the shortest distance between the endpoints of $P$.
\end{lemma}

\begin{proof}
    Without loss of generality, we take $M(P)=(M_0,0,\ldots,0)$.
    Consider each path $P^\prime$ between the endpoints of $P$.
    Because $P$ is a pareto
    shortest path, $P^\prime \not \prec P$. This implies that either $M(P) = M(P^\prime)$ or
    $\sum_{l>0}M^\prime_l > 0$. In the first case, no weighting can yield $dist_W(P^\prime)<dist_W(P)$.
    Otherwise we consider the path $P^\prime$ with $\sum_{l>0}M^\prime_l>0$ minimal, and denote this
    quantity $\sigma$. The weighting $W=(1/M_0,2/\sigma,\ldots,2/\sigma)$ then gives
    $dist_W(P)=1$ and $dist_W(P^\prime)\geq 2$ for $M(P)\not = M(P^\prime)$, proving the lemma.
\end{proof}

The {\bf pareto backbone} is obtained by eliminating each edge that whose multidistance is not among the
the pareto distance set between its parent and child nodes. We denote this $\mathcal{B}(\mathcal{G})$.
This generalizes the concept of a graph
backbone of a graph $G$, which eliminates all edges that are not part of a shortest path and is denoted
$B(G)$.
In fact, the pareto backbone
of a multigraph with layers $\left(G_0,G_1,\ldots,G_m\right)$ is the multigraph whose layers are the
graph backbones of each layer, as we now formally state and prove.

\begin{theorem}
    The pareto backbone $\mathcal{B}(\mathcal{G})$ of a multigraph $\mathcal{G}$ with layers\\
    $\left(G_0,G_1,\ldots,G_m\right)$ is the multigraph $\mathcal{G}^\prime$ with layers
    $\left(B(G_0),B(G_1),\ldots,B(G_m)\right)$.
\end{theorem}
\begin{proof}
    The proof proceeds in two parts.

    ($\mathcal{G}^\prime\subseteq \mathcal{B}(\mathcal{G})$) Consider an edge $(u,v)\in B(G_l)$.
    By definition, the edge $(u,v)$ is a shortest path between $u$ and $v$ in $G_l$ and is of length
    $w((u,v))$. The multidistance in $\mathcal{G}$ along the path $P$ consisting of only this edge is
    zero in all entries except entry $l$, in which it is $w((u,v))$. If a path $P^\prime$ connecting
    $u$ and $v$ in $\mathcal{G}$ satisfies $P^\prime \prec P$, then its $l^{th}$ entry must be less
    than $w((u,v))$ and all other entries must be zero. This implies that $P^\prime$ is a path confined
    to $G_l$ that has a lower weight than $w((u,v))$, contradicting the assumption $(u,v)\in B(G_l)$.

    ($\mathcal{B}(\mathcal{G}) \subseteq \mathcal{G}^\prime$) Consider $(u,v)$ in
    $\mathcal{B}(\mathcal{G})$. By definition, $M((u,v))$ is a pareto shortest path between $u$ and $v$.
    Because this path is of length one, it is confined to a single layer, and thus is a shortest path
    within that layer. Because the edge is a shortest path in its layer, it is part of that layer's
    backbone.
\end{proof}

A corollary of this result relates the pareto backbone and pareto distance set between two nodes.

\begin{corollary}
    The pareto distance set between nodes $u$ and $v$ in $\mathcal{G}$ is equal to the pareto distance
    set between $u$ and $v$ in $\mathcal{B}(\mathcal{G})$.
\end{corollary}
\begin{proof}
    Denote the pareto distance set computed in $\mathcal{G}$ by $D$ and the pareto distance set computed
    in $\mathcal{B}(\mathcal{G})$ by $D^\prime$. 
    
    ($D^\prime\subseteq D$) Because $\mathcal{B}(\mathcal{G})$ is a subgraph of
    $\mathcal{G}$, it follows that every path in $\mathcal{B}(\mathcal{G})$ is also in $\mathcal{G}$, and
    therefore $D^\prime$ is a subset of $D$.

    ($D\subseteq D^\prime$) Consider a path $P$ connecting $u$ to $v$ in $\mathcal{G}$ whose
    multidistance is in $D$. Because $D^\prime\subseteq D$, $M(P)$ is incomparable to
    all multidistances in $D^\prime$. Consider an arbitrary edge of $P$, $(P_i,P_{i+1})$
    in a layer $G_l$. Because $P$ is pareto minimal, there is no path in $G_l$ from $P_i$ to
    $P_{i+1}$ that is shorter than $w((P_i,P_{i+1}))$, and therefore $(P_i,P_{i+1})$ is in the backbone
    of $G_l$. By the previous theorem, this implies that $(P_i,P_{i+1})$ is an edge of 
    $\mathcal{B}(\mathcal{G})$. Applying this argument ot all edges of $P$ reveals that $P$ is a path in
    $\mathcal{B}(\mathcal{G})$. Because $P$ is a path in $\mathcal{B}(\mathcal{G})$ and $M(P)$ is 
    incomparable to
    all multidistances in $D^\prime$, $M(P)$ is an element of $D^\prime$. As this argument applies for 
    arbitrary paths with multidistance in $D^\prime$, it follows that $D$ is a subset of
    $D^\prime$.
\end{proof}

This theorem and its corollary have significant computational implications. They imply that a pareto
backbone and its associated pareto distance sets can be computed from the backbones of individual layers.

\end{document}
